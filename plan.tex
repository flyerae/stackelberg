% 10 points on A4 paper
\documentclass[10pt,a4paper]{report}

% University guidelines for A4
\usepackage{a4-mancs}

% Fonts for postscript printers
\usepackage{pslatex}

% Colours
\usepackage[table,usenames,dvipsnames]{xcolor}

% Gantt Charts
\usepackage{pgfgantt}

% setup type
\renewcommand{\familydefault}{\sfdefault}
\setlength{\parskip}{8pt}
\setlength{\parindent}{0pt}
\linespread{1.2}

% Document title
\title{Team Brogram: Stackelberg Plan}

% Author
\author{Freddy Kelly, Danyal Prout \& Sam Bell}

% Begin document
\begin{document}

% Add title
\maketitle

% Add table of contents
\tableofcontents

% start new page
\newpage

\section{Design}

\subsection{Overview}
\begin{enumerate}
  \item On game intialisation, parse \emph{CSV} and perform batch regression to find follower's reaction function, $R_F(x)$.
  \item For the first day, find global maxima of $J_L[]$ to obtain price to submit.
  \item On proceeding to a new day, take previous follower's price and perform recursive regression to efficiently update approximation of $R(x)$.
  \item Again, find maxima of updated $R(x)$ and submit price. Repeat for each new day.
\end{enumerate}

% Something else here..

\newpage
\section{Schedule}
This section describes each task in more detail.

\subsection{Tasks}
We have broken development down into the following key deliverables:

\begin{enumerate}
  \item Learning the reaction function (\emph{Sam} \& \emph{Danyal})
    \begin{itemize}
      \item Currently assuming the follower's reaction function is linear, so simply representing the function as two variables, $a$ and $b$, from $R(x) = a + bx$.
      \item We then parse CSV data files to obtain historical data on follower responses
      \item After, we perform linear regression via least-squares on this data to to find values for $a$ and $b$.
      \item Regression performed using formula from Xiao-Jun's fourth lecture, slide 20.
      \item Include formula here?
      \item Our next task is to find the global maxima of the function.
    \end{itemize}
  \item Finding the global maxima (\emph{Freddy})
    \begin{itemize}
      \item Having estimated the follower's reaction function, $R(x),$ we will then calculate our optimal strategy by maximising the (leader's) payoff function, $J_L[]$.
    \end{itemize}
\end{enumerate}

\subsection{Gantt Chart}
The Gantt chart below shows how the development will progress over the coming weeks.

% empty f-s labels
\setganttlinklabel{f-s}{}
\begin{ganttchart}%
  [%x unit=1.0cm,
  y unit title=0.4cm,
  y unit chart=0.75cm,
  vgrid,
  title/.style={draw=none, fill=MidnightBlue!50!black!80}, title label font=\sffamily\bfseries\color{white}, title label anchor/.style={below=-1.6ex},
  %title left shift=.05,
  %title right shift=-.05,
  title height=1,
  bar/.style={draw=none, fill=YellowGreen!75},
  bar height=.6,
  bar label font=\normalsize\color{black!50},
  milestone/.style={fill=red, draw=black, rounded corners=1pt},
  group right shift=0,
  group top shift=.6,
  group height=.3,
  group peaks={}{}{.2}, incomplete/.style={fill=Apricot}]{20}
\gantttitle{Schedule (Weeks 7-11)}{20} \\
\gantttitlelist{7,...,11}{4} \\
  \ganttset{progress label text={}, link/.style={black, -to}}
  \ganttbar[progress=100, name=T0]{Planning \& Specification}{1}{1} \\
  \ganttlinkedbar[progress=100, name=T1, link type=dr]{Platform Implementation}{2}{3} \\
  \ganttbar[progress=50, name=T2]{On-line Learning of $R(x)$}{4}{12} \\
  \ganttbar[progress=50, name=T3]{Maximise $J_L[]$}{4}{12} \\
  \ganttmilestone{v1.0 Completion}{12} \\
  \ganttbar[progress=0, name=T4]{Test \& Evaluate Performance}{13}{14} \\
  \ganttbar[progress=0, name=T5]{Least-squares Approach}{15}{20} \\
  \ganttbar[progress=0, name=T6]{Multi-variate Regression}{15}{20} \\
  \ganttmilestone{v2.0 Completion}{20} \\
  \ganttlink[link mid=.5, link type=f-s]{T1}{T2}
  \ganttlink[link mid=.5, link type=f-s]{T1}{T3}
  \ganttlink[link mid=.5, link type=f-s]{T4}{T5}
  \ganttlink[link mid=.5, link type=f-s]{T4}{T6}
\end{ganttchart}

\small
Milestones/deliverables are marked by a \textbf{\color{red} red} diamond.
\normalsize

% End document
\end{document}
